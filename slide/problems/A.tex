\frame
{
  \frametitle{A 游戏 {by \itshape Itst}}

	给定一棵 $n$ 个节点的树,双方在树上博弈.

	初始 \texttt{1} 号节点有一个棋子。先手每次禁掉一条边,后手每次选择一条没被禁掉的边将棋子沿着它移动。后手移到度数恰好为 \texttt{1} 的节点时后手获胜,不能移动时先手获胜.

	问双方绝顶聪明时谁获胜.

	$1\le n\le 10^5$

}

\begin{frame}{题解}
	首先需要注意到 $n=1$ 时先手胜,因为没有节点度数是 \texttt1. \pause

	我们先获得一些比较直观的想法:

	\begin{itemize}
	\item 对于后手来说,走回头路是不优的,因为这样给了先手更多的机会禁边。所以我们可以把树看成以 1 为根的外向树,后手只会往远离 1 的方向移动棋子. \pause
	\item 对于先手来说,每次禁与棋子相连的一条边一定是更优的. \pause
	\item 如果按照后手的策略每次只会往远离 1 的方向走,那么先手每次也一定会禁掉当前棋子所在节点和一个儿子的边.
	\end{itemize}

\end{frame}

\begin{frame}{题解}

	将这个问题看成外向树之后,自然产生了递归的结构: \pause

	\begin{itemize}
	\item 如果有 $\ge 2$ 个子树是获胜的,那么后手直接钻进任意一个获胜的子树内就可以获胜; \pause
	\item 而如果只有 $\le 1$ 个子树获胜,那么先手把这个子树禁掉,后手就没救了. \pause
	\end{itemize}

	而将外向树换成一般的无向树不会改变胜负情况: \pause

	\begin{itemize}
	\item 我们只需要考虑先手获胜的情况会不会发生改变. \pause
	\item 因为每个子树后手都会输,所以后手就算能回头钻进别的子树也没法赢.\pause
	\end{itemize}

	因此 DFS 计算以 \texttt{1} 为根时每个子树的获胜情况即可. \pause

	复杂度 $O(n)$,可以轻松通过.

\end{frame}