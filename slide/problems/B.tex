\frame
{
  \frametitle{B 替换 {by \itshape Itst}}

	给定字符串 $s$,字符集为 \texttt{01}?. 对于每个 $1\le k\le |s|$,定义字符串 $t_k$ 为以下 \texttt{01} 字符串:
	\begin{itemize}
	\item 如果 $s_i$ 不是 ?,则 $t_{k,i}=s_i$.
	\item 否则 $t_{k,i}=t_{k,i-k}$,你需要递归地求出 $t_{k,i-k}$;如果 $i-k$ 越界则为 \texttt{0}.
	\end{itemize}

	你需要输出每个 $t_k$ 中的 \texttt{1} 的数量.

	$1\le |s|\le 10^5$.

}

\frame
{
  \frametitle{题解}

	按照下标顺序得到字符串 $t_k$ 的每个字符,得到 $O(n^2)$ 做法. \pause

	注意到字符集是 \texttt{01},故考虑压位,每次求出 $t_k$ 的 $\omega=64$ 个字符. \pause

	$k\le \omega$ 的情况暴力,$k\ge \omega$ 的情况 $t_{k,i}\dots t_{k,i+\omega}$ 依赖的字符为 $t_{k,i-k}\dots t_{k,i-k+\omega}$,这是提前求好的. 因此我们需要做以下操作得到这段字符: \pause

	\begin{itemize}
	\item 得到 $t_{k,i-k}\dots t_{k,i-k+\omega}$ 的二进制表示; \pause
	\item 得到 $s_{k,i}\dots s_{k,i+\omega}$ 中 ? 的位置和 \texttt{1} 的位置的二进制表示. \pause
	\item 对于 ? 的位置使用 $t_{k,i-k}\dots t_{k,i-k+\omega}$,对于非 ? 的位置使用 $s_{k,i}\dots s_{k,i+\omega}$,这容易在得到二进制表示后使用位运算实现.
	\end{itemize}

}

\frame
{
  \frametitle{题解}

	如果使用一个 long long 存储形如 $t_{k,j\omega}\dots t_{k,(j+1)\omega-1}$ 的一段字符,那么第一个部分只与两个 long long 有关; 如果每次我们取 $i$ 为 $\omega$ 的倍数,那么第二个部分可以直接预处理. \pause

	这样得到复杂度为 $O\left(\frac{n^2}{\omega}\right)$ 的做法,应该可以通过.

}
